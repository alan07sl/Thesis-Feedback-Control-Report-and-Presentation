\chapter{Ball movement control - Imperative}
\label{app:ball-movement}

%\hspace*{-\parindent}
\begin{lstlisting}[style=ScalaStyle, caption={Ball movement control}, label={lst:ball-full-app}]
import javafx.application.Application
import javafx.application.Platform.runLater
import javafx.geometry.Pos
import javafx.scene.canvas.{Canvas, GraphicsContext}
import javafx.scene.input.MouseEvent
import javafx.scene.layout.StackPane
import javafx.scene.paint.Color
import javafx.scene.shape.StrokeLineCap
import javafx.scene.{Scene, SnapshotParameters}
import javafx.stage.{Stage, WindowEvent}

import scala.collection.mutable
import scala.language.postfixOps

class BallMovement extends Application {

  var ball $=$ Ball(ballRadius)
  var setpoint $=$ ball.position
  var prevError, integral $=$ (0.0, 0.0)
  val history $=$ new History
  var historyTick $=$ 0

  def pid: Acceleration $=$ {
    val (kp, ki, kd) $=$ (3.0, 0.0001, 80.0)
    val error $=$ setpoint - ball.position
    val derivative $=$ error - prevError

    integral $=$ integral + error
    prevError $=$ error

    (error * kp + integral * ki + derivative * kd) * 0.001
  }

  def update(implicit gc: GraphicsContext): Unit $=$ {
    val acceleration $=$ pid map (a $\Rightarrow$ math.max(math.min(a, 0.2), -0.2))
    ball $=$ ball accelerate acceleration

    // managing the history
    historyTick $+=$ 1
    if (historyTick $==$ 5) {
      historyTick $=$ 0
      if (history.size $>=$ 50)
        history.dequeue
      history enqueue ball.position
    }

    // drawing all the elements
    runLater(() $\Rightarrow$ Draw.draw(ball.position, setpoint, ball.acceleration, history))
  }

  def start(stage: Stage) $=$ {
    val canvas $=$ new Canvas(width, height)
    val root $=$ new StackPane(canvas)
    root setAlignment Pos.TOP_LEFT
    root.addEventHandler(MouseEvent.MOUSE_CLICKED,
        (e: MouseEvent) $\Rightarrow$ setpoint $=$ (e.getX, e.getY))

    implicit val gc $=$ canvas getGraphicsContext2D
    var running $=$ true
    val loop $=$ new Thread(() $\Rightarrow$ while (running) { update; Thread sleep 16 })

    stage setOnHidden ((_: WindowEvent) $\Rightarrow$ { running $=$ false; loop.join() })
    stage setScene new Scene(root, width, height)
    stage setTitle "Balltracker"
    stage show()

    loop start()
  }
}
object BallMovement extends App {
  Application.launch(classOf[BallMovement])
}
\end{lstlisting}

\begin{lstlisting}[style=ScalaStyle, caption={Ball movement \code{Draw} object}, label={lst:ball-full-draw}]
object Draw {

  def draw(pos: Position, setpoint: Position, acc: Acceleration, history: History)(implicit gc: GraphicsContext) $=$ {
    drawBackground
    drawHistory(history)
    drawLine(pos, setpoint)
    drawSetpoint(setpoint)
    drawBall(pos)
    drawVectors(pos, acc)
  }

  def drawBackground(implicit gc: GraphicsContext) $=$ {
    gc.setFill(Color.rgb(231, 212, 146))
    gc.fillRect(0, 0, width, height)
  }

  def drawBall(point: Position)(implicit gc: GraphicsContext) $=$ {
    val (x, y) $=$ point
    val diameter $=$ 2 * ballRadius

    gc.setFill(Color.rgb(123, 87, 71))
    gc.fillOval(x - ballRadius, y - ballRadius, diameter, diameter)
  }

  def drawSetpoint(setpoint: Position)(implicit gc: GraphicsContext) $=$ {
    val (x, y) $=$ setpoint

    val radius $=$ ballRadius / 4
    val diameter $=$ radius * 2

    gc.setFill(Color.rgb(161, 90, 90))
    gc.fillOval(x - radius, y - radius, diameter, diameter)
  }

  def drawLine(ball: Position, setpoint: Position)(implicit gc: GraphicsContext) $=$ {
    gc.setStroke(Color.rgb(96, 185, 154))
    gc.setLineWidth(1.0)
    gc.setLineDashes(8.0, 14.0)

    gc.beginPath()
    (gc.moveTo _).tupled(setpoint)
    (gc.lineTo _).tupled(ball)
    gc.stroke()
    gc.setLineDashes()
  }

  def drawVectors(pos: Position, acc: Acceleration)(implicit gc: GraphicsContext) $=$ {
    val (px, py) $=$ pos
    val (ax, ay) $=$ acc

    gc.setStroke(Color.rgb(247, 120, 37))
    gc.setLineWidth(8)
    gc.setLineCap(StrokeLineCap.ROUND)

    gc.beginPath()
    gc.moveTo(px, py)
    gc.lineTo(px - ax * 300, py)
    gc.stroke()

    gc.beginPath()
    gc.lineTo(px, py)
    gc.lineTo(px, py - ay * 300)
    gc.stroke()
  }

  def drawHistory(history: History)(implicit gc: GraphicsContext) $=$
    history.synchronized {
      history.zipWithIndex.foreach(item $\Rightarrow$ {
        val ((x, y), index) $=$ item
        val size $=$ history.size
        val alpha $=$ (index: Double) / size

        gc.setFill(Color.rgb(96, 185, 154, alpha))
        gc.fillOval(x, y, 10, 10)
      })
    }
}
\end{lstlisting}

\hspace*{-\parindent}
\begin{lstlisting}[style=ScalaStyle, caption={Ball movement control}, label={lst:ball-full-utils}]
package object ballmovement {

  val ballRadius $=$ 20.0
  val width $=$ 1024
  val height $=$ 768

  type Position $=$ (Double, Double)
  type Velocity $=$ (Double, Double)
  type Acceleration $=$ (Double, Double)
  type History $=$ mutable.Queue[Position]
  
  implicit def toHandler[T $<:$ Event](action: T $\Rightarrow$ Unit): EventHandler[T] $=$
    new EventHandler[T] { override def handle(e: T): Unit $=$ action(e) }

  implicit def toRunnable(runnable: () $\Rightarrow$ Unit): Runnable $=$
    new Runnable { override def run() $=$ runnable() }

  implicit class Tuple2Math[X: Numeric, Y: Numeric](val src: (X, Y)) {
    import Numeric.Implicits._
    def +(other: (X, Y)) $=$ (src._1 + other._1, src._2 + other._2)
    def -(other: (X, Y)) $=$ (src._1 - other._1, src._2 - other._2)
    def *(scalar: X)(implicit ev: Y $=:=$ X) $=$ (scalar * src._1, scalar * src._2)
    def map[Z](f: X $\Rightarrow$ Z)(implicit ev: Y $=:=$ X): (Z, Z) $=$ (f(src._1), f(src._2))
  }

  case class Ball(acc: Acceleration, vel: Velocity, pos: Position) {
    def accelerate(newAcc: Acceleration) $=$ Ball(newAcc, vel + newAcc, pos + vel + newAcc)
  }
  object Ball {
    def apply(radius: Double) $=$ Ball((0.0, 0.0), (0.0, 0.0), (radius, radius))
  }
}
\end{lstlisting}
