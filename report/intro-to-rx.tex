\section{Reactive Extensions}
There have been many attempts to fit the philosophy of reactive programming into libraries, APIs or even languages. \todo{cite some of these: RxMobile, Reactive Extensions, ``Spicing up Dart'', Facebook Rx???, Elm, ReactiveStreams, Akka, \ldots} In this section, we will discuss briefly discuss some of the features of one of these libraries, namely Reactive Extensions (a.k.a. Rx). This project started at Microsoft with an implementation in C\# \todo{cite Erik \@ Microsoft} (Rx.Net), was later ported to Java, Scala, Groovy, JavaScript, \todo{and others???} by NetFlix with the help of the open-source community and has ever since been ported into dozens of other languages.

% Reference to ``Erik @ Microsoft'' is:
% Meijer, E. 2010. Subject/Observer is dual to iterator. Presented at FIT: Fun Ideas and Thoughts at the Conference on Programming Language Design and Implementation; http://www.cs.stanford.edu/pldi10/fit.html

However, these translations have deviated a lot from the original implementation. Most remarkable is that some of them are not even purely `reactive' anymore \todo{reference to talk by Erik Meijer about derivation of Rx}. Given these deviations from the original paradigm and the state of complexity of these implementations, we decided to use a reference implementation of the original Rx that was recently written in Scala by Erik Meijer with the purpose of creating a light-weight implementation for developing mobile apps on Android, called RxMobile. \todo{is there a reference to RxMobile?} The following discussion and derivation of the API will however apply to both Reactive Extensions and RxMobile and in this section we will therefore refer to both as `Rx'.

\subsection{Core components}
Rx is a library for composing asynchronous and event based (reactive) programs by using observable sequences. The core of Rx consists of two interfaces: \obs and \obv. The latter can subscribe and react to the events that are emitted by the former. An \obs can emit zero or more events (called \textit{onNext}) and can possible end with a \textit{onCompleted} or an \textit{onError}. After either a \textit{onCompleted} or an \textit{onError} is emitted, no more events can follow. Therefore the emission protocol can be summarized by the following regular expression: \code{onNext* (onError | onCompleted)?}. \todo{reference to Rx Design Guidelines - http://go.microsoft.com/fwlink/?LinkID=205219} When an \obv subscribes to an \obs,  it will return a \subs. With this one can later unsubscribe from the \obs and clean up potential resources.

\autoref{lst:obs-obv} shows these basic concepts of the \obs, \obv and \subs translated in Scala. Notice that here \subs is a superclass of \obv. Therefore there is no need for the \obs to return a \subs when an \obv subscribes to it. It will however return a \subs when another variant of \code{subscribe} is used, where for example a lambda expression is expected.

\begin{lstlisting}[caption={Observable, Observer and Subscription}, label={lst:obs-obv}]
trait Observable[T] {
    def subscribe(observer: Observer[T]): Unit
    def subscribe(onNext: T $\Rightarrow$ Unit): Subscription
    // other variants of subscribe
}

trait Observer[T] extends Subscription {
    def onNext(t: T): Unit
    def onError(e: Throwable): Unit
    def onCompleted(): Unit
}

trait Subscription {
    def isUnsubscribed(): Boolean
    def unsubscribe(): Unit
}
\end{lstlisting}

Creating an \obs is done by the \code{Observable.create(Observer $\Rightarrow$ Unit): Observable} method, that takes a function from \obv to \code{Unit} and returns an \obs. The input function is then used in the implementation of \code{subscribe}, when a \emph{real} \obv is provided. The \obv can be created by supplying it three lambda expressions, one for each kind of event.

\autoref{lst:create-sub-obs} provides a simple example of how both an \obs and \obv are created and used in practice. Here the function in \code{Observable.create} causes the \obs to emit three values and then complete. Notice that these are only emitted after line~\ref{line:subscribe} is executed, when the \obv is subscribed to the \obs! If no one will subscribe, the values will never be produced nor emitted.

\begin{lstlisting}[caption={Creating and subscribing to an \obs}, label={lst:create-sub-obs}]
val xs: Observable[Int] $=$ Observable.create((obv: Observer[Int]) $\Rightarrow$ {
    obv.onNext(1)
    obv.onNext(2)
    obv.onNext(3)
    obv.onCompleted()
})
val observer: Observer[Int] $=$ Observer((x: Int) $\Rightarrow$ print(x + " "),
    (e: Throwable) $\Rightarrow$ print(e),
    () $\Rightarrow$ print("completed"))

xs.subscribe(observer) |\label{line:subscribe}|

// result: 1 2 3 completed
\end{lstlisting}

Using \code{Observable.create} is a very powerful tool to create an \obs. Many other methods can be derived from it. For example, the \obs in \autoref{lst:create-sub-obs} is often written as \code{Observable.apply(1, 2, 3)}\footnote{Notice that in Scala this can be shortened to \code{Observable(1, 2, 3)}. Explicitly writing \code{.apply} is only done for later referral.}. This way of writing is not only more concise and conveys what the true meaning of this expression is in a better way, but it is also exactly the same, since \code{Observable.apply} is implemented in terms of \code{Observable.create}. In fact, all methods that are defined on \obs can be implemented using \code{Observable.create}!

\subsection{Derivation of \obs and \obv}
In 1994, the book `Design Patterns: Elements of Reusable Object-Oriented Software' by the \textit{Gang of Four} was published \todo{reference}. This book explored the capabilities and pitfalls of object oriented programming and contained an overview of 23 classical software design patterns. Also, the book described the relationships between these 23 design patterns.

One of these design patterns is called the \textit{Observer} pattern and forms the basis of the \obs and \obv interfaces described in the previous subsection. Even though the Gang of Four did identify a lot of relations between the different design patterns, it failed to identify any relation between the Observer pattern and any other pattern, except for the Mediator pattern.

In 2010, Erik Meijer published a short paper called `Subject/Observer is Dual to Iterator' \todo{reference}, where he described a mathematical relationship between the Observer pattern and the Iterator pattern based on categorical duality. The paper shows that instances of the Observer pattern can be viewed as push-based collections, rather than the pull-based collections that result from the Iterator pattern. For later parts of this thesis, it is important to understand the mathematical basis of this relationship between the \obs and \obv interfaces and the \itb and \itr interfaces.

The Iterator pattern consists of two interfaces, called \itb and \itr. (see \autoref{lst:itb-itr}) In most common languages \itb forms the basis of every Collections API. It has only one method \code{iterator} that returns the \itr to iterate over the elements in the collection. The \itr interface on the other hand contains two methods to be implemented: \code{moveNext} and \code{current}. The former performs a side effect by moving a pointer to the next element in the iteration and then returns a \code{Boolean} to indicate whether or not there was a next element. The latter is a pure function that just returns the element the pointer is currently pointing to\footnote{The \itr interface can be written in different ways. For example, Java provides two methods \code{hasNext} and \code{next}. Here the former is considered to be a pure function and the latter performs the side-effect of moving the pointer.}. Notice that the \code{moveNext} method can throw an exception rather than returning \code{false} in case an error occurs.

\begin{lstlisting}[caption={\itb and \itr interfaces}, label={lst:itb-itr}]
trait Iterable[T] {
    def iterator(): Iterator[T]
}
trait Iterator[T] {
    def moveNext(): Boolean // throws Exception
    def current: T
}
\end{lstlisting}

These two interfaces together form the basis of all pull-based or interactive collections. The user asks for the next element and will get one in case a next element can be produced. In the following we will transform these interfaces into push-based or reactive collection, where the user subscribes to a collection and receives data once it is produced. This derivation, as well as its conclusion that interactive and reactive collections are each other's dual, are based on the papers \todo{reference to YMIAD and Erik \@ Microsoft}, as well as several keynotes and Channel9 video's \todo{references}. This derivation, as well as it's intermediate steps are important for later parts of this thesis.

The first step in this derivation is to rewrite the two methods in the \itr interface into a single method \code{getNext()}. Using the categorical \textit{coproduct} \todo{reference to Wikipedia or another source} we can combine these two methods and determine its type signature: \code{getNext()} can either fail with an exception or succeed with either an element or no element, resulting in the type signature \code{getNext(): Try[Option[T]]}. The new, intermediate, set of interfaces is shown in \autoref{lst:itb-itr-interm}.

\begin{lstlisting}[caption={\itr interface after applying coproduct}, label={lst:itb-itr-interm}]
trait Iterable[T] {
    def iterator(): Iterator[T]
}
trait Iterator[T] {
    def getNext(): Try[Option[T]]
}
\end{lstlisting}

Since both interfaces now only have one single method, and since the only purpose of \itb is to produce an \itr, they can be written as a single lambda expression. An \itb can be written as:

\begin{equation} \label{eq:itb}
\code{() $\Rightarrow$ (() $\Rightarrow$ Try[Option[T]])}
\end{equation}

Notice that applying \code{Unit} to the outer lambda yields another lambda expression, which corresponds to the interface for \itr in \autoref{lst:itb-itr-interm}: \code{() $\Rightarrow$ Try[Option[T]]}.

The next step this transformation is to dualize this lambda expression\todo{reference to Wikipedia or another source}. A very informal way of describing duality is to flip all the arrows and rewrite the lambda expression. For example, the duality of $f :: A \rightarrow B$ is $\bar{f} :: A \leftarrow B \equiv B \rightarrow A$. In the same way, we can apply this to lambda~expression~\ref{eq:itb}, resulting in

\begin{equation} \label{eq:obs}
\code{(Try[Option[T]] $\Rightarrow$ ()) $\Rightarrow$ ()}
\end{equation}

This lambda expression takes a lambda from \code{Try[Option[T]]} to \code{Unit}, and returns \code{Unit}.

We can now put this lambda expression back into context by splitting it into two interfaces. The inner lambda can be rewritten to an interface called \obv, which has one method \code{onNext(t: Try[Option[T]]): Unit}. This method can then be further rewritten into three separate methods by expanding the \code{Try[Option[T]]} type: \code{onNext(t: T): Unit}, \code{onError(e: Throwable): Unit} and \code{onCompleted(): Unit}. The outer lambda on the other hand translates to an interface called \obs, which has one method \code{subscribe(obv: Observer[T]): Unit}. Notice how these interfaces are completely identical to the ones presented in \autoref{lst:obs-obv}.

This derivation shows that interactive, pull-based collections are the mathematical dual of reactive, push-based collections. The \obs and \obv interfaces can directly be derived from the \itb and \itr interfaces. Both sets of interfaces can therefore be considered to be collections. In other words: streaming data behaves exactly the same way as regular collections, such as arrays, lists and sets, except for them being push-based rather than pull-based. \todo{cite Erik \@ Microsoft AND YMIAD} In the world of push-based collections one \emph{subscribes} to the stream in order to \emph{react} to the next element that is being send, whereas one \emph{asks} for the next element in a pull-based scenario.

\subsection{Operators}
Besides this difference between push-based and pull-based collections, all other rules apply for both of them. In regular pull-based collections many operators are defined to manipulate, transform, filter, fold or group items. These operators can therefore also be applied to push-based collections. However, rather than iterating over the collection and applying the transformation on each element, these operators \emph{react} to data being emitted by applying their particular transformation or side effect and passing the (transformed) data down to either a potential next operator or the \code{subscribe} method.

The Rx implementations of the \obs interface provide a wide variety of operators that apply all sorts of transformations to a data stream. All operators are defined on \obs and will also return an \obs, making the API highly compositional. In order to understand how these operators work, we will look at some basic examples. Other, more advanced operators will be discussed in section \todo{reference to section on backpressure operators}.

\paragraph{Filter}To select only those elements that satisfy a certain predicate, the operator \code{filter(p: T $\Rightarrow$ Boolean): Observable[T]} is used. Every time an element is received by this operator, the predicate \code{p} will be applied. If the element satisfies the predicate, it is passed downstream; otherwise the element will be discarded. \autoref{lst:operators-obs} shows in line~\ref{line:filter} how to select the odd numbers in a stream of integers by supplying a predicate.

\paragraph{Map}To transform one stream of data into another, the \code{map(f: T $\Rightarrow$ S): Observable[S]} is used. Each time an element (which is of type \code{T}) is received by this operator, the function \code{f} is applied to this element, yielding a new element of type \code{S}. This new element is then passed to down the stream. In \autoref{lst:operators-obs} the \code{map} operator is first applied in line~\ref{line:map} to the stream of filtered elements with a function that doubles the input.

\paragraph{Scan}Most operators do not allow for any form of internal state. They do not keep track of previous elements. An operator that can take the previous elements into account is \code{scan(seed: S)(acc: (S, T) => S): Observable[S]}. To this operator first of all a seed is supplied, which is the internal state of the operator before any value is received. Once an element is received, it will apply its internal state, together with that element to the accumulator function \code{acc} and produce an element to be emitted. This emitted value is also the new internal state of the operator. \autoref{lst:operators-obs} has a \code{scan} operator in line~\ref{line:scan} that takes the sum of all integers it receives and uses a \code{seed = 0}.

\paragraph{Drop}The \code{scan} operator is often used together with \code{drop(n: Int): Observable[T]}, which discards the first \code{n} elements and forwards all elements after that. The combination with the \code{scan} operator is used to prevent the seed value from being emitted further downstream, as is shown in \autoref{lst:operators-obs} line~\ref{line:drop}.

\paragraph{Take}Whereas \code{drop} discards the first \code{n} elements, \code{take(n: Int): Observable[T]} is used to only propagate the first \code{n} elements and discard all elements that come after that. In practice this means that the stream is terminated early with a call to \code{Observer.onCompleted()}. \autoref{lst:operators-obs} shows how \code{take} is used to only propagate the first and the second element and discard the third.

\begin{lstlisting}[caption={Operators on \obs}, label={lst:operators-obs}, columns=fixed]
Observable(1, 2, 3, 4, 5)		// emits:    1, 2, 3, 4, 5
    .filter(x $\Rightarrow$ x $\%$ 2 $==$ 1)			// emits:    1,    3,    5 |\label{line:filter}|
    .map(x $\Rightarrow$ x * 2)			// emits:    2,    6,    10 |\label{line:map}|
    .scan(0)((sum, x) => sum + x)	// emits: 0, 2,    8,    18 |\label{line:scan}|
    .drop(1)				// emits:    2,    8,    18 |\label{line:drop}|
    .take(2)				// emits:    2,    8 |\label{line:take}|
    .subscribe(x $\Rightarrow$ println(x))
\end{lstlisting}

Just as the interactive collections, Rx has defined its operators in a way that composition of operators is very easy. In this way, simple operators can be chained in order to create the complex behavior that is often desired. There are many more operators defined on \obs, which are not mentioned in this section. For a full overview, we refer to the documentation on the Rx websites \todo{reference to reactivex.io, Rx.Net, etc.}. Some other operators will be discussed in \todo{reference to section on backpressure operators}. 

\subsection{Cold vs. Hot}
\todo{write a short piece about the difference between a cold and hot \obs and discussing several kinds of observables, such as mouse/keyboard, database, clocks, `normal', etc.}
\todo{not sure anymore why we needed this subsection. only add it when it turns out we need it!}

\subsection{Subjects}
\todo{a short story about the \code{Subject} and \code{BehaviorSubject}, since we gonna need these in the feedback control section}