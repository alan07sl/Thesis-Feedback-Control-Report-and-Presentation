\chapter{Solving overproduction with feedback control}
\label{chap:solving-overproduction}

In \Cref{chap:exploring-the-problem-space} we discussed how streams can originate from various kinds of sources that can be categorized into three groups. We introduced the \textit{hot} source as a strictly reactive collection of data: there is now way to interact with the stream or control how fast it produces its data. On the contrary, a \textit{cold asynchronous} source can be interacted with, as it has an interface from which one can get zero or more elements. However, as the data to be returned can take some time to be computed, this source is still bound to the same notion of time as is the case with the hot source. Finally there is the group of \textit{cold synchronous} sources, which takes away the notion of time: elements that are requested will be returned immediately.

We also discussed several solutions to overproduction in the light of these three groups of sources. We learned that \textit{avoiding} by grouping or dropping data works perfectly for hot and cold asynchronous sources as a first line of defense. \textit{Callstack blocking} on the other hand is something that is automatically done to cold synchronous sources but can potentially be dangerous to hot and cold asynchronous sources as they might form a buffer of calls on the stack. The \textit{Reactive Streams} solution and RxJava's \textit{reactive pull} are to be used on cold sources alone, and cannot work with a hot source as they go against the contract of reactiveness as defined in \cite{berry1991-Reactive}.

The central problem here is that we want a single reactive interface to share between all kinds of data streams. Although one might argue that you ought not to be using a reactive interface for an interactive (cold) source, we acknowledge the fact that in many circumstances it is more practical to view and treat them as `streaming' and `real-time' data rather than having them as interactive sources. In order to do so, we need a way to interact with cold sources in an overproduction-safe way. Reactive Streams and reactive pull achieve this by introducing the concept of backpressure and changing the reactive interface itself, making the consumer in charge, rather than the producer. Not only is this against the concept of reactiveness, it also gives many problems with implementing the operators defined on the reactive interface.

In this chapter we will propose an alternative to backpressure that makes use of the feedback systems described in \Cref{chap:intro-to-feedback-control,chap:feedback-api}. As we already concluded that backpressure is not suitable for hot sources, we will discard these from the discussion in this chapter. The solution proposed here will apply to interactive sources alone.

\section{Controlling a buffer}
RxJava points out on its wiki page \cite{RxJava-Wiki-Backpressure} about backpressure that it does not make the problem of overproduction in the source go away. It claims to only move ``\textit{the problem up the chain of operators to a point where it can be handled better}''. To do so, they created the \textit{reactive pull} mechanism with operators like \code{onBackpressureBuffer} and \code{onBackpressureDrop}, such that the flow control is moved up to these kinds of operators.

We propose to move this flow control even further up the chain, up to the point where the source of the stream is drained in the pipeline of operators. Only there we can have maximum control over how much data is brought into the stream at a particular point in time. With this we do not have the need for infinite buffers as is the case in \code{onBackpressureBuffer}, nor do we have to drop unprocessed elements as is done with \code{onBackpressureDrop}. We propose to not wrap the cold source in the \code{Observable.create} (or any of its derived factory methods) but to wrap it in a universal, interactive interface. This way we are not dependent in our implementation on what kind of source we are dealing with. Given this interactive interface we can fill a \textit{bounded} buffer with as many elements as can be processed at a particular point in time. The buffer pulls data from the source on behalf of the subscriber, which gets as much data pushed at it as it is able to handle. Pushing an element from the buffer to the downstream will automatically block the thread for another element to be pushed until the first one is fully processed.

To control the buffer's size, we will use feedback control. This makes total sense, as we don't know how fast the downstream is going to drain the buffer. However, it does not make any sense to give a certain size to the setpoint and compare the current size with it, as some `slow' consumers might go faster or slower than expected. Bounding the buffer to a certain fixed size defeats the purpose of the feedback system in this case, as we cannot dynamically grow or shrink the size as needed. On the other hand, it is also not possible to ask ``\textit{make sure the buffer is filled to its optimal size}''. A feedback system is not able to solve this, as it does not have a particular setpoint specified.

Instead of controlling the buffer size directly, we choose to measure the ratio between what goes out the buffer and what comes in the buffer. We will refer to this ratio as the system's throughput. In an optimal situation the amount of data that comes in is just as much as comes out of the system, so ideally this ratio must be $1.0$, which will be the setpoint of this system. Given the error that comes from the difference between the setpoint and the actual throughput, we can then determine how many elements to request from the source in the next iteration. The controller that does this will be discussed in a later section.

The full feedback system is depicted in \todo{refer to diagram}. Here it is also clearly visible that the source itself is \emph{not} part of the feedback system, but is \emph{used} by the system to retrieve a certain number of elements from. Also note that the \textit{downstream handler} is not part of feedback system. Even though it \emph{interacts} with the buffer, it is an external force that influences the behavior of the system. Ultimately the \textit{downstream handler} is the part that exposes an \obs for an \obv to listen to.

\todo{aankondiging: ``in de rest van deze section bespreken we de verschillende facetten van het feedback systeem''}

\todo{diagram of this feedback system}

\subsection{A universal, interactive interface}
As mentioned above, we propose to not wrap the (cold) source directly in \code{Observable.create}, but instead wrap it in a universal, interactive interface. This is necessary since there are many variants of interactive interfaces that all do the same, but each one in a slightly different way.

For example, the \itr interface has an \code{hasNext} and \code{next} method, which respectively check if there is a next element and return the next element. C\#'s \ier on the contrary has methods such as \code{moveNext}, which fetches the next element and returns whether there is a next element, and \code{current}, which actually returns the next element. For SQL database interaction, Java defines a \code{ResultSet}. This interface has a method called \code{next}, which moves the cursor to the next row of the result, and methods such as \code{getInt(int columnIndex)} and \code{getString(int columnIndex)} to get the content of a specific type from a column in the row the cursor is pointing to.

One thing these interfaces have in common is that they contain a method that fetches a single element and in the mean time block the thread it is operating on. If this fetch takes some time, your program will have to wait for the result to come in. To prevent this blocking behavior, we propose a universal interactive interface in which you request an element and subscribe to a stream on which \textit{eventually} this element will be emitted. Note that we separate the concerns of \textit{requesting} a next element and \textit{receiving} a next element. In this way, the program can still continue to operate and maybe do some other things while it is waiting for the requested element.

Given that we will use this interface in a feedback system that controls a buffer, we will pose an extra requirement on this interface. As the feedback system's controller might conclude that $n > 1$ elements need to be requested from the source, we must have to possibility to do so. Rather than $n$ times requesting 1 element, we want to request $n$ elements at once.

The complete interface is called \code{Requestable[T]} and is shown in \Cref{lst:universal-interactive-interface}. It contains a single abstract method \code{request(n: Int): Unit}, which is called whenever the user of this interface wants a certain number of elements from the source. The requested elements will at some point in time be emitted by the \obs that is returned by \code{results: Observable[T]}. If no more elements are available in the source, this \obs will terminate with an \code{onCompleted} event. The implementor of \code{Requestable} is expected to use the \code{subject} to bring elements in the stream, whereas the user of the interface is expected to observe \code{results} in order to get the requested data. Note that this is a \emph{hot} stream: element emission will not be repeated as a second \obv subscribes to the stream.

Example implementations of this interface for \itr and \code{ResultSet} are included in Appendix~\ref{app:backpressure-solution}.

\begin{minipage}{\linewidth}
\begin{lstlisting}[style=ScalaStyle, caption={Universal, interactive interface used in the feedback system}, label={lst:universal-interactive-interface}]
trait Requestable[T] {

  protected final val subject $=$ PublishSubject[T]()

  final def results: Observable[T] $=$ subject

  def request(n: Int): Unit
}
\end{lstlisting}
\end{minipage}

\subsection{The feedback system}
Now that we are able to interact with any cold source via the \code{Requestable} interface, we can continue designing and discussing the actual feedback system that controls the size of the bounded buffer. As stated before, we do not control the \emph{actual} size of the buffer by using a setpoint of any arbitrary, fixed number of elements. Instead we observe how many elements were taken out of the buffer in relation to how many elements were in the buffer during a particular time span. With that we do in fact not control the buffer's \emph{size}, but rather control the \emph{throughput} of the buffer, while making changes to the number of elements that are requested from the source at every feedback cycle.

The throughput in a particular time span ($\tau_t$) is defined in terms of how many elements are there to be consumed in relation to how many of these elements are actually being consumed. In a scenario where the elements that are not consumed in a certain time span are discarded or where the buffer is flushed at the end of each time span, the throughput would be equal to the ratio of how many elements were being consumed to how many elements were presented to be consumed in a certain time span. In our case, however, we do not wish to discard any elements but rather keep the left-over elements from the previous time span and make them part of what is available to be consumed in the next time span. With this we can define the throughput $\tau_t$ at time $t$ as

\begin{equation}\label{eq:throughput-fraction}
\tau_t = \frac{q_{t-1} + n_t - q_t }{q_{t-1} + n_t} \text{ \textbf{with} }q_{t-1}\text{, }q_t\text{, }n_t\text{ integers} \geq 0
\end{equation}

or

\begin{equation}\label{eq:throughput-simple}
\tau_t = 1 - \frac{q_t}{q_{t-1} + n_t} \text{ \textbf{with} }q_{t-1}\text{, }q_t\text{, }n_t\text{ integers} \geq 0
\end{equation}

In these formulas, $q_t$ is the size of the buffer at time $t$, whereas $n_t$ is the number of elements that has been put in the buffer between time $t - 1$ and $t$.

\Cref{eq:throughput-simple} provides us with a sense of the range of $\tau_t$. Since $q_t \leq q_{t-1} + n_t$ (it is not possible to take out more elements than are present in the buffer) we can guarantee a lower bound for $\tau_t$ of $0.0$. Likewise, since $q_{t-1}, q_t, n_t \geq 0$, we can set an upper bound for $\tau_t$ of $1.0$. Still there is the possibility of dividing by 0, but we will guard against this in the next couple of paragraphs.

\begin{equation}\label{eq:range-of-tau}
0.0 \leq \tau_t \leq 1.0
\end{equation}

With $\tau$ as the metric for the feedback system that controls the buffer, it is not difficult to come up with an appropriate setpoint. We want the throughput to be as high as possible, which is, given \Cref{eq:range-of-tau}, $1.0$.

The next point in designing this feedback loop is to determine when a new cycle starts. For this we have to observe that it will only make sense for a new cycle to start if the downstream as polled at least one element from the buffer. If in a certain time span the downstream is too busy processing one element, it does not make any sense to do a new measurement of the throughput. As new elements have been coming in based on the previous feedback cycle, but no elements have been taken out of the buffer, we do not need to request more elements. Instead, we just extend the time span by merging it with the next, until at least one element has been taken out of the buffer. Only then the feedback loop will run a new cycle.

Note that using this definition of a feedback cycle is a guard against dividing by 0 in \Cref{eq:throughput-fraction,eq:throughput-simple}. This can only happen when at the start of a time span the buffer is empty and during this time span no elements are coming into the buffer. This can either be due to an unfortunate decision of the controller (which we will discuss in a later section) the request no further elements from the source, even though the buffer is empty, or because it takes some amount of time before the source can produce its next element. If the buffer was empty at the start and no elements were coming in, the downstream would at no point during this time span be able to poll an element from the buffer. Because of this, the current time span is merged with the next time span, without running through a whole new cycle and therefore also without running into dividing by 0 while calculating $\tau$.



\clearpage
\section*{Content}
\begin{itemize}
	\item[\checkmark] Distinction between hot and cold - we're only considering cold, as hot is not able to handle backpressure
	\item[\checkmark] Bring handling backpressure to the start of the stream
	\item[\checkmark] Feedback control of buffer - general overview of idea + diagram
	\item Requestable interface
		\begin{itemize}
			\item request items and let them come down the stream on their own terms (i.e. in a reactive way)
			\item some implementations of Requestable (iterable collection, Database ResultSet)
		\end{itemize}
	\item Implementation of BackpressureObservable
		\begin{itemize}
			\item feedback system fills the queue
			\item reactively pull data from the queue and send it downstream (exposed as Observable)
		\end{itemize}
	\item Controller
		\begin{itemize}
			\item PID control, why it doesn't work here
			\item relation with Janert
			\item incremental controller
		\end{itemize}
	\item How it works in practice
		\begin{itemize}
			\item backpressure examples, but then with the feedback solution
			\item show graphs of how many items get requested each time
		\end{itemize}
	\item Conclusion
\end{itemize}