\chapter{Ball movement control - Reactive}
\label{app:ball-movement-reactive}

\begin{lstlisting}[style=ScalaStyle, caption={Ball movement control}, label={lst:ball-full-app-reactive}]
import javafx.application.Application
import javafx.rx.Events
import javafx.scene.Scene
import javafx.scene.canvas.Canvas
import javafx.scene.layout.StackPane
import javafx.stage.Stage

import applied_duality.reactive.schedulers.JavaFxScheduler
import fbc.Component
import fbc.commons.Controllers

import scala.concurrent.duration.DurationInt
import scala.language.postfixOps

class BallTracker extends Application {

  val kp $=$ 3.0
  val ki $=$ 0.0001
  val kd $=$ 80.0

  val pidX $=$ Controllers.pidController(kp, ki, kd)
  val pidY $=$ Controllers.pidController(kp, ki, kd)

  def feedback: Component[Position, Ball2D] $=$ {
    val fbcX $=$ Component.create[Position, Pos](_._1) >>> feedbackSystem(pidX)
    val fbcY $=$ Component.create[Position, Pos](_._2) >>> feedbackSystem(pidY)

    fbcX.combine(fbcY)(Ball2D(_, _))
  }

  def feedbackSystem(pid: Component[Double, Acc]): BallFeedbackSystem $=$ {
    pid.map(d $\Rightarrow$ math.max(math.min(d * 0.001, 0.2), -0.2))
      .scan(new AccVel)(_ accelerate _).drop(1)
      .scan(Ball1D(ballRadius))(_ move _)
      .sample(16 milliseconds)
      .feedback(_ position)
  }

  def start(stage: Stage) $=$ {
    val canvas $=$ new Canvas(width, height)
    implicit val gc $=$ canvas.getGraphicsContext2D
    Draw.drawInit

    val root $=$ new StackPane(canvas)
    root setAlignment javafx.geometry.Pos.TOP_LEFT

    val history $=$ new History

    root.mouseClicked
      .map(event $\Rightarrow$ (event.getX, event.getY))
      .publish(clicks $\Rightarrow$ feedback.run(clicks).withLatestFrom(clicks)((_, _)))
      .observeOn(JavaFxScheduler())
      .tee(x $\Rightarrow$ {
        val (ball, goal) $=$ x
        Draw.draw(ball.position, goal, ball.acceleration, history)
      })
      .map(_._1.position)
      .buffer(5)
      .map(_.last)
      .subscribe(pos $\Rightarrow$ {
        history.synchronized {
          if (history.size $>=$ 50)
            history dequeue()
          history enqueue pos
        }
      })

    stage setScene new Scene(root, width, height)
    stage setTitle "Balltracker"
    stage show()
  }
}

object BallTracker extends App {
  Application.launch(classOf[BallTracker])
}
\end{lstlisting}

\begin{lstlisting}[style=ScalaStyle, caption={Ball movement \code{Draw} object}, label={lst:ball-full-draw-reactive}]
import javafx.scene.canvas.GraphicsContext
import javafx.scene.paint.Color
import javafx.scene.shape.StrokeLineCap

object Draw {

  def draw(pos: Position, setpoint: Position, acc: Acceleration, history: History)(implicit gc: GraphicsContext) $=$ {
    drawBackground
    drawHistory(history)
    drawLine(pos, setpoint)
    drawSetpoint(setpoint)
    drawBall(pos)
    drawVectors(pos, acc)
  }

  def drawInit(implicit gc: GraphicsContext) $=$ {
    drawBackground
    drawBall(ballRadius, ballRadius)
  }

  def drawBackground(implicit gc: GraphicsContext) $=$ {
    gc.setFill(Color.rgb(231, 212, 146))
    gc.fillRect(0, 0, width, height)
  }

  def drawBall(point: Position)(implicit gc: GraphicsContext) $=$ {
    val (x, y) $=$ point
    val diameter $=$ 2 * ballRadius

    gc.setFill(Color.rgb(123, 87, 71))
    gc.fillOval(x - ballRadius, y - ballRadius, diameter, diameter)
  }

  def drawSetpoint(setpoint: Position)(implicit gc: GraphicsContext) $=$ {
    val (x, y) $=$ setpoint

    val radius $=$ ballRadius / 4
    val diameter $=$ radius * 2

    gc.setFill(Color.rgb(161, 90, 90))
    gc.fillOval(x - radius, y - radius, diameter, diameter)
  }

  def drawLine(ball: Position, setpoint: Position)(implicit gc: GraphicsContext) $=$ {
    gc.setStroke(Color.rgb(96, 185, 154))
    gc.setLineWidth(1.0)
    gc.setLineDashes(8.0, 14.0)

    gc.beginPath()
    (gc.moveTo _).tupled(setpoint)
    (gc.lineTo _).tupled(ball)
    gc.stroke()
    gc.setLineDashes()
  }

  def drawVectors(pos: Position, acc: Acceleration)(implicit gc: GraphicsContext) $=$ {
    val (px, py) $=$ pos
    val (ax, ay) $=$ acc

    gc.setStroke(Color.rgb(247, 120, 37))
    gc.setLineWidth(8)
    gc.setLineCap(StrokeLineCap.ROUND)

    gc.beginPath()
    gc.moveTo(px, py)
    gc.lineTo(px - ax * 300, py)
    gc.stroke()

    gc.beginPath()
    gc.lineTo(px, py)
    gc.lineTo(px, py - ay * 300)
    gc.stroke()
  }

  def drawHistory(history: History)(implicit gc: GraphicsContext) $=$ {
    history.synchronized {
      history.zipWithIndex.foreach(item $\Rightarrow$ {
        val ((x, y), index) $=$ item
        val size $=$ history.size
        val alpha $=$ (index: Double) / size

        gc.setFill(Color.rgb(96, 185, 154, alpha))
        gc.fillOval(x, y, 10, 10)
      })
    }
  }
}
\end{lstlisting}

\begin{minipage}{\linewidth}
\begin{lstlisting}[style=ScalaStyle, caption={Ball movement control}, label={lst:ball-full-utils-reactive}]
import scala.collection.mutable

package object balltracker {

  val ballRadius $=$ 20.0
  val width $=$ 1024
  val height $=$ 768

  type Pos $=$ Double
  type Vel $=$ Double
  type Acc $=$ Double
  type Position $=$ (Pos, Pos)
  type Velocity $=$ (Vel, Vel)
  type Acceleration $=$ (Acc, Acc)
  type History $=$ mutable.Queue[Position]
  type BallFeedbackSystem $=$ Component[Pos, Ball1D]

  case class AccVel(acceleration: Acc $=$ 0.0, velocity: Vel $=$ 0.0) {
    def accelerate(acc: Acc): AccVel $=$ {
      AccVel(acc, velocity + acc)
    }
  }

  case class Ball1D(acceleration: Acc, velocity: Vel, position: Pos) {
    def move(av: AccVel): Ball1D $=$ {
      Ball1D(av.acceleration, av.velocity, position + av.velocity)
    }
  }
  object Ball1D {
    def apply(position: Pos): Ball1D $=$ Ball1D(0.0, 0.0, position)
  }

  case class Ball2D(acceleration: Acceleration, velocity: Velocity, position: Position)
  object Ball2D {
    def apply(x: Ball1D, y: Ball1D): Ball2D $=$ {
      Ball2D((x.acceleration, y.acceleration), (x.velocity, y.velocity), (x.position, y.position))
    }
  }
}
\end{lstlisting}
\end{minipage}

\begin{minipage}{\linewidth}
\begin{lstlisting}[style=ScalaStyle, caption={JavaFx/Rx interface}, label={lst:ball-full-rx-reactive}]
package object rx {

  implicit def toHandler[T $<:$ Event](action: T $\Rightarrow$ Unit): EventHandler[T] $=$ {
    new EventHandler[T] { override def handle(e: T): Unit $=$ action(e) }
  }

  implicit class Events(val node: Node) extends AnyVal {
    def getEvent[T $<:$ InputEvent](event: EventType[T]): Observable[T] $=$ Observable.create[T](observer $\Rightarrow$ {
      val handler $=$ (e: T) $\Rightarrow$ observer.onNext(e)
      node.addEventHandler(event, handler)
      observer $+=$ Subscription { node.removeEventHandler(event, handler) }
    })

    def mouseClicked: Observable[MouseEvent] $=$ getEvent(MouseEvent.MOUSE_CLICKED)
  }
}
\end{lstlisting}
\end{minipage}
