\section{Reactive Programming}
\label{sec:reactive-programming}
One of the most difficult problems in computer science in this day and age is handling big amounts of data. No longer are applications bound to the closed world of a single machine and a relational database. Applications these days have access the whole World Wide Web, exist of large clusters of machines, work with data ranged from SQL-style relational databases to key-value pointer based databases, as well as binary data such as images, audio and video. Also the speed in which data is handled varies from `once a month' to `every millisecond'.

These changes in how applications need to perform, ask for new ways of handling data. No longer can we allow ourself to load a whole database table into memory for further processing, nor can we permit ourself to wait for all data to be downloaded before we start processing it. We need systems and concepts that can handle data right as it gets pushed to our application, without further delay, preferably in an asynchronous way and without blocking other processes or waiting for all data to have arrived. \cite{meijer2012-YMIAD}

An interesting part of the solution to these problems is Reactive Programming. This term is described by Albert Benveniste and G\'erard Berry in ``\textit{Real time programming: special purpose or general purpose languages}'' \cite{berry1991-Reactive} as he makes a distinction between \textit{interactive} and \textit{reactive} programs:

\begin{quote}
``\textit{Interactive programs} interact at their own speed with users or with other programs; from a user point of view, a time-sharing system is interactive. \textit{Reactive programs} also maintain a continuous interaction with their environment, but \textbf{at a speed which is determined by the environment}, not by the program itself.''
\end{quote}

Reactive programs `observe' events that occur in their environment and react to them as specified by the program. These events can vary from large amounts of data coming in over a network connection or from a database, to mouse moves or other kinds of UI events and from low-latency sensor streams to high-latency calls to web services.

Also notice that Benveniste and Berry explicitly emphasize that that reactive programs run at a speed which is determined by the environment. This means that the producer (which is part of the environment) is in charge of sending data to the consumer. Therefore the consumer (the program) cannot \emph{ask} for new data, it can only \emph{react} to the data that has been sent by the producer. This relationship between producer and consumer is often referred to as \textit{push based behavior}.

A classic example of push based behavior is a mouse pointer moving over the screen. Every time the pointer is moved, it will \emph{push} its new coordinates to the screen, for it to drawn in the new position. From its point of view, the screen \emph{reacts} to the new coordinates being received by the mouse pointer.

This push based behavior is in contrast with the relation between producer and consumer in an interactive program. Here, according to Benveniste and Berry, the consumer (program) interacts on its own speed with its environment. The consumer will determine the speed at which data is transmitted from the outside by continuously asking for the next bit of data. After this is received, the consumer processes the data and then asks for the next piece. This kind of interaction between consumer and producer is often referred to as \textit{pull based behavior}.

One example of a simple interactive program is a foreach-loop iterating over a collection of elements. As long as there are more elements in the collection, it will ask for the next one, process it according to the loop's body and then ask for the next element.